\documentclass[]{scrartcl}
%\documentclass[11pt]{article} 
\usepackage[a4paper,left=2cm,right=2cm,top=2cm,bottom=2cm,bindingoffset=0mm]{geometry}

\usepackage[utf8]{inputenc}
\usepackage[T1]{fontenc}
\usepackage{listings}
\usepackage{fancyref}
\usepackage{graphicx}
\usepackage{float}
\usepackage{hyperref}
\usepackage{url}
\usepackage{wrapfig}
\usepackage{amssymb}
\usepackage{physics}
\usepackage{siunitx}
\usepackage{amsmath}
%\usepackage[onehalfspacing]{setspace}

%\usepackage{subfigure}
\usepackage{subcaption}  
\usepackage[abs]{overpic}
\usepackage{graphicx}
\usepackage{pdfpages} 
\usepackage[ngerman]{datetime}
\usepackage[ngerman]{babel}
\usepackage{grffile}

\setlength{\parindent}{0pt}
\pagestyle{empty}

\title{Analysemethoden in der Hochenergiephysik II: Übungsblatt 5}
\author{Kristina Schmitt}

\begin{document}	
	
	\maketitle
	
	\section*{Aufgabe 1}
	
	\textit{Warum sind die meisten Final-State-Teilchen Photonen?}\\
	--
	
	
	\textit{Rapiditätsverteilungen}\\
	
	\begin{figure}[h]
		\centering
		\includegraphics[width=0.95\linewidth]{rapidity.png}
		\caption[]{Rapiditätsverteilung aller produzierter Teilchen und von Myonen in 10.000 Events}
		\label{fig:rap}
	\end{figure}

	\textit{Warum ist das Rapiditätsspektrum der Myonen schmäler?}\\
	--

	\newpage
	
	\section*{Aufgabe 2}
	
	\textit{Impulse der Jets und der Teilchen, jeweils auf zwei Raumdimesionen projeziert (links und rechts) für verschiedene Rapiditätscuts}\\
	
	\begin{figure*}[h]
		\centering
		\begin{subfigure}[b]{0.7\textwidth}
			\centering
			\includegraphics[width=0.7\textwidth]{jets_rap_low.png} 
			
		\end{subfigure}
		\hfill
		\begin{subfigure}[b]{0.7\textwidth}  
			\centering 
			\includegraphics[width=0.7\textwidth]{jets_default.png}
			
		\end{subfigure}
		\begin{subfigure}[b]{0.7\textwidth}  
			\centering 
			\includegraphics[width=0.7\textwidth]{jets_rap_high.png}
			
		\end{subfigure}
		\caption{Impulse der Jets und der Teilchen, jeweils auf zwei Raumdimensionen projiziert (links und rechts) für verschiedene Rapiditätscuts, von oben nach unten aufsteigend.}
		\label{fig:momenta}
	\end{figure*}

	Je höher der Rapiditätscut, desto mehr Teilchen werden zu einem Jet zusammengefasst. ich schätze, dass sich der Rapiditätscut auf die maximal mögliche räumliche Ausdehnung eines Jets bezieht.\\
	
	\newpage
	
	\textit{Random Number Seed = 0 (time), jeweils 100 Ereignisse generiert.} In Tabelle \ref{tab:num} ist für je 100 Ereignisse die Anzahl an Ereignissen mit mindestens 3 Jets eingetragen. Der Mittelwert ergibt sich zu 24,1. \\
	
	\begin{table}[h]
		\centering
	\begin{tabular}{|c|c|c|c|c|c|c|c|c|c|}
		\hline 
		22 & 26 & 19 & 17 & 18 & 34 & 27 & 27 & 23 & 25 \\ 
		\hline 
		21 & 26 & 23 & 22 & 26 & 22 & 26 & 26 & 26 & 26 \\ 
		\hline 
	\end{tabular} 
	\caption{Anzahl Ereignisse mit mehr als 3 Jets aus jeweils 100 Ereignissen.}
	\label{tab:num}
	\end{table}
	
	
\end{document}